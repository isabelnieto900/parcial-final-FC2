\documentclass[11pt,a4paper]{article}
\usepackage[utf8]{inputenc}
\usepackage[T1]{fontenc}
\usepackage[spanish]{babel}
\usepackage{amsmath, amssymb, amsfonts}
\usepackage{graphicx}
\usepackage{geometry}
\geometry{a4paper, margin=2.5cm}
\usepackage{hyperref}
\hypersetup{
    colorlinks=true,
    linkcolor=blue,
    filecolor=magenta,
    urlcolor=cyan,
}
\usepackage{listings}
\usepackage{caption}
\usepackage{subcaption}
\usepackage{siunitx}
\usepackage{physics} % Para notación física como \vb (vector bold), \expval, \pdv

\title{Parte A: Simulación del Movimiento Browniano en un Medio Viscoso}
\author{Física Computacional II - Grupo [Número/Nombre del Grupo]}
\date{\today}

\begin{document}
\maketitle

\section{Objetivo}
El objetivo principal de esta parte del proyecto es simular el movimiento browniano de una partícula inmersa en un medio viscoso. Se busca analizar su trayectoria y estudiar propiedades estadísticas como el desplazamiento cuadrático medio para verificar la relación de difusión de Einstein.

\section{Fundamento Físico}
El movimiento de una partícula browniana de masa $m$ puede describirse mediante la ecuación de Langevin:
\begin{equation}
    m\frac{d\vb{v}}{dt} = -\gamma\vb{v} + \vb{\eta}(t)
    \label{eq:langevin}
\end{equation}
donde:
\begin{itemize}
    \item $\vb{v}$ es la velocidad de la partícula.
    \item $-\gamma\vb{v}$ es la fuerza de arrastre viscoso, con $\gamma$ siendo el coeficiente de fricción. Para una partícula esférica de radio $R$ en un fluido de viscosidad $\eta_f$, $\gamma = 6\pi\eta_f R$.
    \item $\vb{\eta}(t)$ es una fuerza estocástica (aleatoria) que representa las colisiones de las moléculas del fluido con la partícula. Esta fuerza tiene las siguientes propiedades estadísticas:
    \begin{align}
        \expval{\eta_i(t)} &= 0 \\
        \expval{\eta_i(t)\eta_j(t')} &= 2\gamma k_B T\,\delta_{ij}\,\delta(t-t')
        \label{eq:fuerza_estocastica}
    \end{align}
    donde $k_B$ es la constante de Boltzmann y $T$ es la temperatura absoluta del fluido.
\end{itemize}
La relación de Einstein para la difusión establece que el desplazamiento cuadrático medio $\expval{\vb{r}(t) - \vb{r}(0)}^2$ de la partícula es proporcional al tiempo para tiempos suficientemente largos:
\begin{equation}
    \expval{(\Delta\vb{r}(t))^2} = 2dDt
    \label{eq:difusion_einstein}
\end{equation}
donde $d$ es la dimensionalidad del movimiento y $D$ es el coeficiente de difusión, dado por $D = \frac{k_B T}{\gamma}$.

\section{Implementación Numérica}
\subsection{Clase ParticulaBrowniana}
Se implementó una clase \texttt{ParticulaBrowniana} para encapsular las propiedades y el comportamiento de la partícula. Sus miembros privados incluyen la posición $\vb{r}$, la velocidad $\vb{V}$, la masa $m$, el coeficiente de fricción $\gamma$, y el producto $k_B T$.

\subsection{Método de Euler-Maruyama}
Para integrar numéricamente la ecuación de Langevin (Ecuación \ref{eq:langevin}), que es una ecuación diferencial estocástica, se utilizó el método de Euler-Maruyama. La discretización de la ecuación es:
\begin{align}
    \vb{V}_{n+1} &= \vb{V}_n - \frac{\gamma}{m}\vb{V}_n \Delta t + \frac{1}{m}\sqrt{2\gamma k_B T \Delta t} \vb{N}(0,1) \\
    \vb{r}_{n+1} &= \vb{r}_n + \vb{V}_{n+1} \Delta t
\end{align}
donde $\vb{N}(0,1)$ es un vector de números aleatorios independientes extraídos de una distribución normal estándar (media cero, varianza uno) y $\Delta t$ es el paso de tiempo. El término $\sqrt{\Delta t} \vb{N}(0,1)$ representa el incremento de Wiener $d\vb{W}$.

\section{Resultados y Análisis}
\subsection{Parámetros de Simulación}
\begin{itemize}
    \item Masa de la partícula ($m$): \SI{1e-15}{\kilo\gram}
    \item Radio de la partícula ($R$): \SI{1}{\micro\meter}
    \item Temperatura ($T$): \SI{300}{\kelvin}
    \item Viscosidad del medio ($\eta_f$, agua): \SI{0.85e-3}{\pascal\second}
    \item Coeficiente de fricción ($\gamma = 6\pi\eta_f R$): Calculado.
    \item Paso de tiempo ($\Delta t$): \SI{0.001}{\second}
    \item Tiempo total de simulación: \SI{10}{\second}
\end{itemize}

\subsection{Trayectoria de la Partícula}
Se simuló la trayectoria de una partícula en 2D (o 3D proyectada).
% Incluir figura de la trayectoria
\begin{figure}[h!]
    \centering
    % \includegraphics[width=0.7\textwidth]{../results/trayectoria_browniana.png}
    \caption{Ejemplo de trayectoria 2D de una partícula browniana.}
    \label{fig:trayectoria}
\end{figure}

\subsection{Desplazamiento Cuadrático Medio (MSD)}
Se calculó el MSD promediando sobre múltiples trayectorias (o una trayectoria muy larga, asumiendo ergodicidad para ciertos regímenes).
% Incluir figura del MSD vs. tiempo
\begin{figure}[h!]
    \centering
    % \includegraphics[width=0.7\textwidth]{../results/msd_browniano.png}
    \caption{Desplazamiento cuadrático medio en función del tiempo. La línea recta indica el comportamiento difusivo esperado.}
    \label{fig:msd}
\end{figure}
A partir de la pendiente de la gráfica MSD vs. $t$, se puede estimar el coeficiente de difusión $D$.

\section{Conclusiones (Parte A)}
Breve resumen de los hallazgos y la validación del modelo.

% \section{Apéndice: Código}
% \lstinputlisting[language=C++, caption=ParticulaBrowniana.h, basicstyle=\footnotesize\ttfamily]{../../include/ParticulaBrowniana.h}
% \lstinputlisting[language=C++, caption=SimuladorBrowniano.cpp, basicstyle=\footnotesize\ttfamily]{../../src/SimuladorBrowniano.cpp}

\end{document}
