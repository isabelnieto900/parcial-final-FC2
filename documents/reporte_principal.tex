\documentclass[11pt,a4paper]{article}
\usepackage[utf8]{inputenc}
\usepackage[T1]{fontenc}
\usepackage[spanish]{babel}
\usepackage{amsmath, amssymb, amsfonts}
\usepackage{graphicx}
\usepackage{geometry}
\geometry{a4paper, margin=2.5cm}
\usepackage{hyperref}
\hypersetup{
    colorlinks=true,
    linkcolor=blue,
    filecolor=magenta,
    urlcolor=cyan,
    pdftitle={Proyecto Final Física Computacional II},
    pdfpagemode=FullScreen,
}
\usepackage{enumitem}
\usepackage{xcolor}
\usepackage{tcolorbox}
\usepackage{listings} % Para incluir código
\usepackage{caption}  % Para captions en figuras y tablas
\usepackage{subcaption} % Para subfiguras
\usepackage{siunitx} % Para unidades SI
\usepackage{physics} % Para notación física

\title{\textbf{Proyecto Final de Física Computacional II}}
\author{Nombre del Grupo \\ \texttt{integrante1@email.com} \\ \texttt{integrante2@email.com}}
\date{\today}

\begin{document}
\maketitle
\begin{abstract}
Este documento presenta el desarrollo del proyecto final para el curso de Física Computacional II. El proyecto abarca la simulación de sistemas físicos mediante programación orientada a objetos, la generación y análisis de números aleatorios, y la aplicación del método de Monte Carlo a problemas de física estadística. Se detallan las implementaciones en C++, los métodos numéricos empleados, los resultados obtenidos y su análisis físico correspondiente.
\end{abstract}

\tableofcontents
\newpage

% ----------------------------------------------------------------------
% Aquí se incluirían los contenidos de los otros archivos .tex
% o se escribiría directamente el contenido.
% Por ejemplo, podrías tener:
% \documentclass[11pt,a4paper]{article}
\usepackage[utf8]{inputenc}
\usepackage[T1]{fontenc}
\usepackage[spanish]{babel}
\usepackage{amsmath, amssymb, amsfonts}
\usepackage{graphicx}
\usepackage{geometry}
\geometry{a4paper, margin=2.5cm}
\usepackage{hyperref}
\hypersetup{
    colorlinks=true,
    linkcolor=blue,
    filecolor=magenta,
    urlcolor=cyan,
}
\usepackage{listings}
\usepackage{caption}
\usepackage{subcaption}
\usepackage{siunitx}
\usepackage{physics} % Para notación física como \vb (vector bold), \expval, \pdv

\title{Parte A: Simulación del Movimiento Browniano en un Medio Viscoso}
\author{Física Computacional II - Grupo [Número/Nombre del Grupo]}
\date{\today}

\begin{document}
\maketitle

\section{Objetivo}
El objetivo principal de esta parte del proyecto es simular el movimiento browniano de una partícula inmersa en un medio viscoso. Se busca analizar su trayectoria y estudiar propiedades estadísticas como el desplazamiento cuadrático medio para verificar la relación de difusión de Einstein.

\section{Fundamento Físico}
El movimiento de una partícula browniana de masa $m$ puede describirse mediante la ecuación de Langevin:
\begin{equation}
    m\frac{d\vb{v}}{dt} = -\gamma\vb{v} + \vb{\eta}(t)
    \label{eq:langevin}
\end{equation}
donde:
\begin{itemize}
    \item $\vb{v}$ es la velocidad de la partícula.
    \item $-\gamma\vb{v}$ es la fuerza de arrastre viscoso, con $\gamma$ siendo el coeficiente de fricción. Para una partícula esférica de radio $R$ en un fluido de viscosidad $\eta_f$, $\gamma = 6\pi\eta_f R$.
    \item $\vb{\eta}(t)$ es una fuerza estocástica (aleatoria) que representa las colisiones de las moléculas del fluido con la partícula. Esta fuerza tiene las siguientes propiedades estadísticas:
    \begin{align}
        \expval{\eta_i(t)} &= 0 \\
        \expval{\eta_i(t)\eta_j(t')} &= 2\gamma k_B T\,\delta_{ij}\,\delta(t-t')
        \label{eq:fuerza_estocastica}
    \end{align}
    donde $k_B$ es la constante de Boltzmann y $T$ es la temperatura absoluta del fluido.
\end{itemize}
La relación de Einstein para la difusión establece que el desplazamiento cuadrático medio $\expval{\vb{r}(t) - \vb{r}(0)}^2$ de la partícula es proporcional al tiempo para tiempos suficientemente largos:
\begin{equation}
    \expval{(\Delta\vb{r}(t))^2} = 2dDt
    \label{eq:difusion_einstein}
\end{equation}
donde $d$ es la dimensionalidad del movimiento y $D$ es el coeficiente de difusión, dado por $D = \frac{k_B T}{\gamma}$.

\section{Implementación Numérica}
\subsection{Clase ParticulaBrowniana}
Se implementó una clase \texttt{ParticulaBrowniana} para encapsular las propiedades y el comportamiento de la partícula. Sus miembros privados incluyen la posición $\vb{r}$, la velocidad $\vb{V}$, la masa $m$, el coeficiente de fricción $\gamma$, y el producto $k_B T$.

\subsection{Método de Euler-Maruyama}
Para integrar numéricamente la ecuación de Langevin (Ecuación \ref{eq:langevin}), que es una ecuación diferencial estocástica, se utilizó el método de Euler-Maruyama. La discretización de la ecuación es:
\begin{align}
    \vb{V}_{n+1} &= \vb{V}_n - \frac{\gamma}{m}\vb{V}_n \Delta t + \frac{1}{m}\sqrt{2\gamma k_B T \Delta t} \vb{N}(0,1) \\
    \vb{r}_{n+1} &= \vb{r}_n + \vb{V}_{n+1} \Delta t
\end{align}
donde $\vb{N}(0,1)$ es un vector de números aleatorios independientes extraídos de una distribución normal estándar (media cero, varianza uno) y $\Delta t$ es el paso de tiempo. El término $\sqrt{\Delta t} \vb{N}(0,1)$ representa el incremento de Wiener $d\vb{W}$.

\section{Resultados y Análisis}
\subsection{Parámetros de Simulación}
\begin{itemize}
    \item Masa de la partícula ($m$): \SI{1e-15}{\kilo\gram}
    \item Radio de la partícula ($R$): \SI{1}{\micro\meter}
    \item Temperatura ($T$): \SI{300}{\kelvin}
    \item Viscosidad del medio ($\eta_f$, agua): \SI{0.85e-3}{\pascal\second}
    \item Coeficiente de fricción ($\gamma = 6\pi\eta_f R$): Calculado.
    \item Paso de tiempo ($\Delta t$): \SI{0.001}{\second}
    \item Tiempo total de simulación: \SI{10}{\second}
\end{itemize}

\subsection{Trayectoria de la Partícula}
Se simuló la trayectoria de una partícula en 2D (o 3D proyectada).
% Incluir figura de la trayectoria
\begin{figure}[h!]
    \centering
    % \includegraphics[width=0.7\textwidth]{../results/trayectoria_browniana.png}
    \caption{Ejemplo de trayectoria 2D de una partícula browniana.}
    \label{fig:trayectoria}
\end{figure}

\subsection{Desplazamiento Cuadrático Medio (MSD)}
Se calculó el MSD promediando sobre múltiples trayectorias (o una trayectoria muy larga, asumiendo ergodicidad para ciertos regímenes).
% Incluir figura del MSD vs. tiempo
\begin{figure}[h!]
    \centering
    % \includegraphics[width=0.7\textwidth]{../results/msd_browniano.png}
    \caption{Desplazamiento cuadrático medio en función del tiempo. La línea recta indica el comportamiento difusivo esperado.}
    \label{fig:msd}
\end{figure}
A partir de la pendiente de la gráfica MSD vs. $t$, se puede estimar el coeficiente de difusión $D$.

\section{Conclusiones (Parte A)}
Breve resumen de los hallazgos y la validación del modelo.

% \section{Apéndice: Código}
% \lstinputlisting[language=C++, caption=ParticulaBrowniana.h, basicstyle=\footnotesize\ttfamily]{../../include/ParticulaBrowniana.h}
% \lstinputlisting[language=C++, caption=SimuladorBrowniano.cpp, basicstyle=\footnotesize\ttfamily]{../../src/SimuladorBrowniano.cpp}

\end{document}

% \documentclass[11pt,a4paper]{article}
\usepackage[utf8]{inputenc}
\usepackage[T1]{fontenc}
\usepackage[spanish]{babel}
\usepackage{amsmath, amssymb, amsfonts}
\usepackage{graphicx}
\usepackage{geometry}
\geometry{a4paper, margin=2.5cm}
\usepackage{hyperref}
\hypersetup{
    colorlinks=true,
    linkcolor=blue,
    filecolor=magenta,
    urlcolor=cyan,
}
\usepackage{enumitem}
\usepackage{xcolor}
\usepackage{listings} % Para incluir código
\usepackage{caption}

\title{Investigación: Generación de Números Aleatorios y Caminatas Aleatorias}
\author{Física Computacional II - Grupo [Número/Nombre del Grupo]}
\date{\today}

\begin{document}
\maketitle

\section{Números Pseudoaleatorios}
\subsection{Concepto y Requisitos}
Los números pseudoaleatorios (PRNG por sus siglas en inglés, Pseudo-Random Number Generators) son secuencias de números generadas por algoritmos determinísticos que aparentan ser aleatorias. Aunque no son verdaderamente aleatorios (ya que, dada la semilla inicial, la secuencia es reproducible), pueden pasar diversas pruebas estadísticas de aleatoriedad.

Requisitos clave para un buen PRNG:
\begin{itemize}
    \item \textbf{Periodo largo:} La secuencia debe ser muy larga antes de repetirse.
    \item \textbf{Uniformidad:} Los números deben estar distribuidos uniformemente en el rango deseado (e.g., [0,1)).
    \item \textbf{Independencia:} Los números sucesivos deben parecer independientes entre sí (baja correlación).
    \item \textbf{Eficiencia computacional:} Deben poder generarse rápidamente.
    \item \textbf{Reproducibilidad:} Dada la misma semilla, se debe generar la misma secuencia (importante para depuración y verificación).
    \item \textbf{Robustez estadística:} Deben pasar un conjunto amplio de pruebas estadísticas de aleatoriedad.
\end{itemize}

\subsection{Ejemplo de Generador Simple en C++ y Correlaciones}
Un generador congruencial lineal (LCG) es un ejemplo simple:
$X_{n+1} = (a X_n + c) \pmod m$
donde $X_n$ es el n-ésimo número, $a$ es el multiplicador, $c$ el incremento, y $m$ el módulo.
Los números normalizados se obtienen como $U_n = X_n / m$.

\begin{lstlisting}[language=C++, caption=Ejemplo de LCG simple, basicstyle=\footnotesize\ttfamily]
#include <iostream>
#include <vector>
#include <fstream>

class LCG {
private:
    unsigned long long seed;
    unsigned long long a; // multiplicador
    unsigned long long c; // incremento
    unsigned long long m; // modulo

public:
    LCG(unsigned long long seed_val, unsigned long long mult,
        unsigned long long incr, unsigned long long modu)
        : seed(seed_val), a(mult), c(incr), m(modu) {}

    double random_uniform() {
        seed = (a * seed + c) % m;
        return static_cast<double>(seed) / m;
    }
};

int main() {
    // Parámetros de Numerical Recipes (no necesariamente los mejores, solo ejemplo)
    LCG generador(12345, 1664525, 1013904223, 4294967296);

    std::ofstream outfile("results/lcg_output.dat");
    std::ofstream outfile_corr("results/lcg_correlation.dat");

    double u_anterior = generador.random_uniform();
    outfile << u_anterior << std::endl;

    for (int i = 0; i < 10000; ++i) {
        double u_actual = generador.random_uniform();
        outfile << u_actual << std::endl;
        outfile_corr << u_anterior << " " << u_actual << std::endl;
        u_anterior = u_actual;
    }
    outfile.close();
    outfile_corr.close();
    return 0;
}
\end{lstlisting}
La visualización de correlaciones se puede hacer graficando $U_n$ vs $U_{n+1}$. Para un buen generador, los puntos deberían llenar el cuadrado unitario de forma uniforme, sin estructuras aparentes. Los LCG simples a menudo muestran estructuras de hiperplanos.

\subsection{Uso y Precauciones de los RNG en Simulaciones Físicas}
Los RNG son cruciales en simulaciones que involucran procesos estocásticos (Monte Carlo, movimiento Browniano, etc.).
Precauciones:
\begin{itemize}
    \item \textbf{Calidad del generador:} Usar generadores probados y de alta calidad. Evitar \texttt{rand()} de C estándar para simulaciones serias debido a su corto periodo y posibles correlaciones.
    \item \textbf{Semilla (Seed):} Siempre inicializar (sembrar) el generador. Para reproducibilidad, usar una semilla fija. Para corridas "diferentes", usar una semilla basada en el tiempo o \texttt{std::random\_device}.
    \item \textbf{Correlaciones:} Tener cuidado con las correlaciones, especialmente en dimensiones altas. Generadores deficientes pueden introducir artefactos.
    \item \textbf{Periodo:} Asegurarse de que el periodo del generador sea mucho mayor que el número de números aleatorios necesarios para la simulación.
    \item \textbf{Múltiples streams:} Si se necesitan múltiples secuencias independientes (e.g., en paralelo), usar técnicas adecuadas para generar sub-secuencias o generadores con parámetros diferentes y probados.
\end{itemize}

\subsection{Descripción del Generador MIXMAX}
MIXMAX (Matrix Recursive Maximum Period Generator) es una familia de generadores de números pseudoaleatorios desarrollados por G. Savvidy y K. Savvidy. Se caracteriza por:
\begin{itemize}
    \item \textbf{Periodo extremadamente largo:} Puede alcanzar periodos del orden de $10^{4000}$ o más, dependiendo de la dimensión de la matriz utilizada.
    \item \textbf{Buenas propiedades estadísticas:} Ha pasado rigurosas baterías de pruebas como TestU01.
    \item \textbf{Basado en recursión matricial:} La generación de números se basa en una operación matricial sobre un vector de estado, módulo un primo grande. La matriz tiene propiedades especiales que garantizan el periodo largo y buenas propiedades de mezcla.
    \item \textbf{Eficiencia:} Puede ser muy eficiente, especialmente en versiones optimizadas.
    \item \textbf{Teoría sólida:} Su diseño se basa en principios de sistemas dinámicos y teoría de números.
\end{itemize}
Es considerado uno de los generadores de más alta calidad disponibles actualmente para simulaciones científicas exigentes.

\section{Caminatas Aleatorias (Random Walks)}
\subsection{Definición y Relación con Difusión}
Una caminata aleatoria es un proceso matemático que describe una trayectoria consistente en una sucesión de pasos aleatorios en algún espacio (e.g., una retícula).
\begin{itemize}
    \item \textbf{Simple (SRW):} En cada paso, el caminante elige una dirección al azar con igual probabilidad entre las opciones disponibles.
    \item \textbf{Relación con difusión:} El desplazamiento cuadrático medio $\expval{R_N^2}$ de una caminata aleatoria simple de $N$ pasos de longitud $l$ en $d$ dimensiones es típicamente $\expval{R_N^2} \propto Nl^2$. Si cada paso toma un tiempo $\Delta t$, entonces el tiempo total es $t = N\Delta t$. Así, $\expval{R_t^2} \propto t$, que es la firma de un proceso difusivo. El movimiento Browniano es un ejemplo físico de un proceso modelado por una caminata aleatoria en el límite continuo.
\end{itemize}

\subsection{Comparación de Generadores en C++}
\begin{itemize}
    \item \texttt{rand()} (\texttt{<cstdlib>}): Parte del estándar C. Generalmente no recomendado para simulaciones serias.
        \begin{itemize}
            \item Pros: Simple de usar, universalmente disponible.
            \item Contras: Calidad a menudo pobre (LCG simple), periodo corto (e.g., $2^{31}-1$ para \texttt{RAND\_MAX}), puede tener correlaciones significativas. No es thread-safe sin precauciones. La distribución no está garantizada como uniforme en todos los bits.
        \end{itemize}
    \item \texttt{drand48()} (familia, \texttt{<cstdlib>} en sistemas POSIX):
        \begin{itemize}
            \item Pros: Mejor calidad que \texttt{rand()} en muchos sistemas, periodo más largo ($2^{48}$). Genera dobles en [0,1).
            \item Contras: No es parte del estándar C++, disponibilidad depende del sistema (POSIX). Puede tener problemas de thread-safety.
        \end{itemize}
    \item Biblioteca \texttt{<random>} (C++11 y posterior): Proporciona un framework mucho más robusto y flexible.
        \begin{itemize}
            \item Pros:
                \begin{itemize}
                    \item Múltiples motores de generación (e.g., \texttt{std::mt19937} - Mersenne Twister, \texttt{std::ranlux48}).
                    \item Periodos muy largos (e.g., $2^{19937}-1$ para MT19937).
                    \item Excelentes propiedades estadísticas para motores como MT19937.
                    \item Separación entre motores (generan bits "crudos") y distribuciones (e.g., \texttt{std::uniform\_int\_distribution}, \texttt{std::normal\_distribution}, \texttt{std::exponential\_distribution}).
                    \item Permite control de semillas y reproducibilidad. Objetos generadores pueden ser instanciados localmente, mejorando la gestión en programas complejos y multihilo.
                \end{itemize}
            \item Contras: Ligeramente más complejo de configurar inicialmente que un simple \texttt{rand()}. El rendimiento puede variar entre motores, pero los buenos suelen ser eficientes.
        \end{itemize}
\end{itemize}
\textbf{Recomendación:} Para simulaciones físicas en C++, siempre se debe preferir la biblioteca \texttt{<random>} sobre \texttt{rand()} o \texttt{drand48()}. El generador \texttt{std::mt19937} es una excelente opción de propósito general.

\end{document}

% \documentclass[11pt,a4paper]{article}
\usepackage[utf8]{inputenc}
\usepackage[T1]{fontenc}
\usepackage[spanish]{babel}
\usepackage{amsmath, amssymb, amsfonts}
\usepackage{graphicx}
\usepackage{geometry}
\geometry{a4paper, margin=2.5cm}
\usepackage{hyperref}
\hypersetup{
    colorlinks=true,
    linkcolor=blue,
    filecolor=magenta,
    urlcolor=cyan,
}
\usepackage{enumitem}
\usepackage{xcolor}
\usepackage{listings} % Para incluir código
\usepackage{caption}
\usepackage{siunitx}
\usepackage{physics}

\title{Investigación Final: Método de Monte Carlo y Física Estadística}
\author{Física Computacional II - Grupo [Número/Nombre del Grupo]}
\date{\today}

\begin{document}
\maketitle

\section{Introducción General al Método de Monte Carlo}
El método de Monte Carlo (MC) es una amplia clase de algoritmos computacionales que se basan en el muestreo aleatorio repetido para obtener resultados numéricos. Su característica esencial es el uso de la aleatoriedad para resolver problemas que podrían ser determinísticos en principio. Los métodos MC son particularmente útiles para:
\begin{itemize}
    \item Simular sistemas con muchos grados de libertad acoplados (como en física estadística).
    \item Calcular integrales multidimensionales complejas.
    \item Optimizar funciones en espacios grandes.
    \item Modelar fenómenos donde la aleatoriedad es intrínseca (e.g., decaimiento radiactivo, difusión).
\end{itemize}
El nombre proviene del Casino de Monte Carlo, debido a la naturaleza aleatoria del proceso, similar a los juegos de azar.

\section{Tipos de Integrales y Problemas que Resuelve}
El método de Monte Carlo es muy poderoso para la integración numérica, especialmente en dimensiones altas.
La idea básica para una integral unidimensional $\displaystyle I = \int_a^b f(x) dx$ es reescribirla como un valor esperado:
\[ I = (b-a) \int_a^b f(x) \frac{1}{b-a} dx = (b-a) \expval{f(X)} \]
donde $X$ es una variable aleatoria distribuida uniformemente en $[a,b]$.
El valor de la integral se estima promediando $f(x_i)$ sobre $N$ muestras $x_i$ tomadas de la distribución uniforme:
\[ I \approx (b-a) \frac{1}{N} \sum_{i=1}^N f(x_i) \]
El error de esta estimación típicamente disminuye como $1/\sqrt{N}$, independientemente de la dimensionalidad de la integral, lo que hace a MC ventajoso sobre métodos de cuadratura tradicionales (e.g., regla del trapecio, Simpson) para integrales de alta dimensión.

Problemas que resuelve:
\begin{itemize}
    \item \textbf{Integración multidimensional:} Calcular $\displaystyle \int \dots \int f(x_1, \dots, x_d) dx_1 \dots dx_d$.
    \item \textbf{Estimación de volúmenes:} Por ejemplo, el cálculo de $\pi$ contando puntos dentro de un círculo inscrito en un cuadrado.
    \item \textbf{Simulación de procesos estocásticos.}
    \item \textbf{Mecánica Estadística:} Cálculo de promedios termodinámicos en diferentes ensambles.
\end{itemize}
Técnicas más avanzadas incluyen el \emph{muestreo por importancia} (importance sampling), donde las muestras se toman de una distribución que se asemeja al integrando para reducir la varianza.

\section{Aplicaciones a Distintos Ensambles en Física Estadística}
\subsection{Ensamble Microcanónico (NVE)}
En el ensamble microcanónico, el sistema aislado tiene un número fijo de partículas ($N$), volumen ($V$) y energía ($E$). Todas las configuraciones microscópicas accesibles con esa energía son igualmente probables.
Los métodos MC pueden usarse para explorar el espacio de fases con energía constante, aunque es menos directo que en el canónico. Algoritmos como el de Creutz (demonio de Monte Carlo) pueden simular el NVE. Se calculan promedios como:
\[ \expval{A} = \frac{1}{\Omega(E)} \sum_{\text{estados } i \text{ con } E_i=E} A_i \]
donde $\Omega(E)$ es el número de estados con energía $E$.

\subsection{Ensamble Canónico (NVT)}
Sistema en contacto con un baño térmico a temperatura $T$ fija, con $N$ y $V$ fijos. La probabilidad de un estado $i$ con energía $E_i$ es $P_i = \frac{e^{-\beta E_i}}{Z}$, donde $\beta = 1/(k_B T)$ y $Z = \sum_i e^{-\beta E_i}$ es la función de partición canónica.
El algoritmo de Metropolis es fundamental aquí:
\begin{enumerate}
    \item Empezar con una configuración inicial.
    \item Proponer un cambio pequeño (e.g., mover una partícula, invertir un espín).
    \item Calcular el cambio de energía $\Delta E$.
    \item Si $\Delta E < 0$, aceptar el cambio.
    \item Si $\Delta E \ge 0$, aceptar el cambio con probabilidad $e^{-\beta \Delta E}$.
\end{enumerate}
Este proceso genera una secuencia de estados distribuidos según la probabilidad de Boltzmann, permitiendo calcular promedios termodinámicos:
\[ \expval{A} = \sum_i A_i P_i \approx \frac{1}{M} \sum_{k=1}^M A_k \]
donde $A_k$ son los valores de la observable $A$ en los $M$ estados generados por la cadena de Markov.

\subsection{Ensamble Gran Canónico ($\mu$VT)}
Sistema en contacto con un baño térmico y un reservorio de partículas, con potencial químico $\mu$, volumen $V$ y temperatura $T$ fijos. El número de partículas $N$ puede fluctuar.
La probabilidad de un estado $i$ con energía $E_i$ y $N_i$ partículas es $P_i = \frac{e^{-\beta (E_i - \mu N_i)}}{\mathcal{Z}}$, donde $\mathcal{Z}$ es la gran función de partición.
Los algoritmos MC para el ensamble gran canónico incluyen, además de los movimientos del ensamble canónico, pasos que intentan cambiar el número de partículas (e.g., inserción o eliminación de partículas), aceptados con probabilidades que dependen de $\mu$.

\section{Aplicación Específica: Integración de $e^{-x^2}$ por Muestreo Aleatorio}
Este es el problema B2.4, y será la implementación central para esta sección de investigación, según las instrucciones actualizadas.

El objetivo es calcular la integral:
\[ I = \int_0^1 e^{-x^2}\,dx \]
utilizando el método de Monte Carlo por muestreo simple. El valor de esta integral está relacionado con la función error $\text{erf}(x)$ a través de la identidad $\int_0^z e^{-t^2}dt = \frac{\sqrt{\pi}}{2}\text{erf}(z)$.
Para $z=1$, el valor aproximado es $0.7468241328$.

\subsection{Método}
Se generan $N$ números aleatorios $x_i$ distribuidos uniformemente en el intervalo $[0,1]$. La estimación de la integral es:
\[ I \approx \frac{1}{N} \sum_{i=1}^N e^{-x_i^2} \]
El error de esta estimación se espera que disminuya como $1/\sqrt{N}$.

\subsection{Implementación}
Se desarrolló una clase \texttt{IntegradorMonteCarlo} que toma como entrada la función a integrar y los límites de integración. Esta clase genera los puntos aleatorios y calcula la integral y una estimación del error. El código fuente se encuentra en \texttt{ParteB2/include/IntegradorMonteCarlo.h} y \texttt{ParteB2/src/IntegradorMonteCarlo.cpp}. El programa principal para esta parte es \texttt{ParteB2/src/main\_montecarlo\_integral.cpp}.

\subsection{Resultados Esperados}
Se espera generar una gráfica que muestre el valor estimado de la integral y el error estimado en función del número de muestras $N$. Esta gráfica debería ilustrar la convergencia del método hacia el valor teórico y la disminución del error. Los datos para esta gráfica se guardarán en \texttt{ParteB2/results/integral\_error.dat}.

% Aquí se incluiría la gráfica una vez generada:
% \begin{figure}[h!]
%     \centering
%     % \includegraphics[width=0.8\textwidth]{../ParteB2/results/integral_error_vs_N.png}
%     \caption{Convergencia del valor de la integral y del error estimado en función del número de muestras $N$.}
%     \label{fig:integral_error}
% \end{figure}

\section{Conclusiones (Investigación Final)}
El método de Monte Carlo proporciona una herramienta versátil y potente para la integración numérica, especialmente útil en casos donde los métodos analíticos son intratables o las dimensiones son altas. La implementación para la integral de $e^{-x^2}$ demuestra la convergencia del método y la naturaleza estadística del error. Aunque se simplificó el alcance de esta sección de investigación para enfocarse en una sola aplicación implementada, los principios generales del método de Monte Carlo son ampliamente aplicables en diversos problemas de la física estadística y otras áreas científicas.

\end{document}

% ----------------------------------------------------------------------

\section{Introducción General}
Descripción breve del proyecto y su estructura.

\section{Parte A: Simulación de Movimiento Browniano}
\label{sec:parte_a}
Referencia al documento \texttt{browniano.tex} o contenido directo.
\subsection{Objetivo}
Simular el movimiento browniano de una partícula en un medio viscoso y analizar su comportamiento difusivo.
\subsection{Fundamento Físico}
Se describe la ecuación de Langevin:
\[ m\frac{d\vec v}{dt}= -\gamma\vec v + \vec\eta(t) \]
donde $\vec\eta(t)$ es un término de fuerza estocástica con propiedades:
\[ \langle\eta_i(t)\eta_j(t')\rangle=2\gamma k_B T\,\delta_{ij}\,\delta(t-t'). \]
\subsection{Implementación y Resultados}
Detalles de la clase \texttt{ParticulaBrowniana} y el método de Euler–Maruyama. Se presentarán resultados de la simulación, como la trayectoria de la partícula y el desplazamiento cuadrático medio.

\section{Parte B1: Generación de Números Aleatorios y Caminatas Aleatorias}
\label{sec:parte_b1}
Referencia al documento \texttt{investigacion\_aleatorios.tex} o contenido directo.
\subsection{Investigación sobre Números Aleatorios}
Resumen de los conceptos de números pseudoaleatorios, generadores y su uso.
\subsection{Problema B1.2: Caminata Aleatoria Autoevitante (SAW)}
Implementación de SAW en una retícula 2D. Análisis del tiempo de CPU y el máximo $N$ factible. Investigación sobre algoritmos más eficientes.

\section{Parte B2: Aplicación del Método de Monte Carlo}
\label{sec:parte_b2}
\subsection{Problema B2.4: Integración de $e^{-x^2}$}
Cálculo de la integral $\displaystyle\int_0^1 e^{-x^2}\,dx$ usando el método de Monte Carlo. Se graficará el error de la estimación en función del número de muestras.

\section{Investigación Final: Método de Monte Carlo y Física Estadística}
\label{sec:investigacion_montecarlo}
Referencia al documento \texttt{montecarlo\_fisica\_estadistica.tex}.
\subsection{Introducción al Método de Monte Carlo}
Principios generales y aplicaciones.
\subsection{Aplicación Específica: Integración de $e^{-x^2}$}
Esta sección se enfocará en la aplicación del método de Monte Carlo para la integración de $e^{-x^2}$, que también es el problema B2.4.
Se detallará la implementación POO y los resultados. Los otros problemas sencillos listados originalmente en las instrucciones para esta sección de investigación no serán implementados como parte de este entregable, según la clarificación recibida.

\section{Conclusiones}
Resumen de los aprendizajes y resultados principales del proyecto.

\appendix
\section{Código Fuente Principal}
\label{app:codigo}
% Se pueden incluir fragmentos importantes del código usando el paquete listings
% Ejemplo:
% \subsection{Clase Vector3D}
% \lstinputlisting[language=C++, caption=Vector3D.h]{../include/Vector3D.h}

\bibliographystyle{abbrv}
% \bibliography{referencias} % Si se usa un archivo .bib

\end{document}
