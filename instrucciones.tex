\documentclass[11pt]{article}
\usepackage[utf8]{inputenc}
\usepackage[T1]{fontenc}
\usepackage[spanish]{babel}
\usepackage{amsmath, amssymb}
\usepackage{graphicx}
\usepackage{geometry}
\geometry{margin=2cm}
\usepackage{hyperref}
\usepackage{enumitem}
\usepackage{xcolor}
\usepackage{tcolorbox}
\usepackage{siunitx}
\usepackage{physics}

\title{\textbf{Examen Final de Física Computacional II}}
\author{Profesor: John  Díaz }
\date{}

\begin{document}
\maketitle
\tableofcontents
\newpage

%-------------------------------------------------
\section*{Instrucciones Generales}
Cada \textbf{grupo de trabajo} (máximo dos personas) debe desarrollar un proyecto final compuesto por \textbf{tres partes} seleccionadas según el esquema siguiente:

\begin{tcolorbox}[colback=gray!10,colframe=black!80,title=Trabajo por grupo]
\begin{itemize}
  \item \textbf{Un (1) problema de la Parte A} – Simulación de un sistema físico usando Programación Orientada a Objetos (POO).
  \item \textbf{Un (1) problema de la Parte B1} – Generación y análisis de números aleatorios.
  \item \textbf{Un (1) problema de la Parte B2} – Aplicación del método de Monte Carlo a la Física Estadística.
\end{itemize}
\end{tcolorbox}

\noindent Para \emph{todas} las partes se exige:
\begin{itemize}
  \item Implementación en \textbf{C++} con POO y estructura modular: \texttt{src/}, \texttt{include/}, \texttt{results/}, \texttt{scripts/}, \texttt{documents/}, \texttt{bin/}.
  \item Métodos numéricos apropiados (\emph{Verlet}, \emph{Euler}, \emph{RK4}, etc.).
  \item Visualización científica (trayectorias, histogramas, energías, distribuciones).
  \item Documentación completa en \LaTeX\ (teoría, resultados, análisis físico).
  \item Comentarios con formato \texttt{Doxygen} y,si quiere ir más allá, un archivo \texttt{Doxyfile}.
  \item \texttt{Makefile} funcional para compilar el proyecto, generar documentación y producir gráficos/informes.
\end{itemize}

Cada entrega debe incluir:
\begin{enumerate}[label=\alph*)]
  \item Código fuente y ejecutable.
  \item Scripts de graficación (\texttt{.gp}(gnuplot), \texttt{.py}(python), \texttt{.m}(octave)).
  \item Carpeta \texttt{results/} con datos y Figuras.
  \item Informe principal en \LaTeX.
  \item Si quiere: Documentación HTML/PDF generada con \texttt{doxygen}.
  \item Carpeta comprimida con toda la estructura.
\end{enumerate}

%-------------------------------------------------
\section{Parte A: Simulación de Sistemas Físicos con POO}
Este problema se acompaña de objetivos, fundamento físico, requisitos técnicos, entrada/salida, visualización, documentación y criterios de evaluación.
%------------ Problema 6 ------------------------------------------------
\subsection{Problema 6: Movimiento Browniano en Medio Viscoso}

\subsubsection*{Objetivo}
Simular el movimiento browniano y analizar la difusión.

\subsubsection*{Fundamento}
\[
m\frac{d\vec v}{dt}= -\gamma\vec v + \vec\eta(t),\quad
\langle\eta_i(t)\eta_j(t')\rangle=2\gamma k_B T\,\delta_{ij}\,\delta(t-t').
\]

\subsubsection*{POO}
Clase \texttt{ParticulaBrowniana}; método de \emph{Euler–Maruyama}.

\subsubsection*{Visualización \& Documentación}
\texttt{results/browniano.dat}; scripts \texttt{plot\_browniano.*}; informe \texttt{documents/browniano.tex}.


\section{Parte B1: Generación de Números Aleatorios y Caminatas Aleatorias}

Antes de resolver los problemas, cada grupo debe entregar un documento \texttt{documents/investigacion.tex} que explique:
\begin{itemize}
  \item Concepto y requisitos de los \emph{pseudonúmeros aleatorios}.
  \item Ejemplo de generador simple en C++ y visualización de correlaciones.
  \item Uso y precauciones de los RNG en simulaciones físicas.
  \item Descripción del generador \texttt{MIXMAX}.
  \item Definición de \emph{caminata aleatoria} y su relación con difusión.
  \item Comparación de \texttt{rand()}, \texttt{drand48()}, \texttt{<random>}, etc.
\end{itemize}

%------------ B1.2 ------------------------------------------------------
\subsection{Problema B1.2: Caminata Aleatoria Autoevitante (SAW)}

\begin{itemize}
  \item Implementar SAW en retícula 2D; medir tiempo de CPU y máximo $N$ factible.
  \item Investigar un algoritmo más eficiente y describirlo.
  \item Clase sugerida: \texttt{SAWSimulador}.
\end{itemize}


\section{Parte B2: Aplicación del Método de Monte Carlo a la Física Estadística}

Seleccione \textbf{un} problema.


%------------ B2.4 ------------------------------------------------------
\subsection{Problema B2.4: Integración de $e^{-x^2}$ por Muestreo Aleatorio}

\begin{itemize}
  \item Calcular $\displaystyle\int_0^1 e^{-x^2}\,dx$ con Monte Carlo.
  \item Graficar error vs.\ número de muestras.
  \item Clase sugerida: \texttt{IntegradorMonteCarlo}.
\end{itemize}

\section*{Investigación Final: Método de Monte Carlo y Física Estadística}

Entregar \texttt{documents/montecarlo.tex} con:
\begin{itemize}
  \item Introducción general al método de Monte Carlo.
  \item Tipos de integrales y problemas que resuelve.
  \item Aplicaciones a distintos ensambles (microcanónico, canónico, gran canónico).
  \item Propuesta e implementación de \textbf{cinco} problemas sencillos:
    \begin{enumerate}[label=\alph*)]
      \item Cálculo de $\pi$ (disco en cuadrado).
      \item Energía media de un gas ideal 1D.
      \item Lanzamiento de monedas.
      \item Integración de $e^{-x^2}$.
      \item Partición canónica para dos niveles.
    \end{enumerate}
\end{itemize}

El problema debe:
\begin{itemize}
  \item Usar POO y modularización.
  \item Guardar resultados en \texttt{results/}.
  \item Incluir visualizaciones en \texttt{scripts/}.
  \item Documentarse en el mismo archivo \LaTeX.
\end{itemize}

%-------------------------------------------------
\section*{Entrega y Calificación}
\begin{itemize}
  \item \textbf{Fecha límite:9/7/23} .
  \item Puntuación total: \textbf{100 pts}.  Cada parte vale 33.3 pts, ponderados según:
    \begin{itemize}
      \item Video de Sustentación  (20\%).  
      \item Correctitud física y numérica (30\%).
      \item Estructura de código y POO (20\%).
      \item Visualización y análisis (20\%).
      \item Documentación y estilo (\LaTeX\ + Doxygen) (10\%).
    \end{itemize}
  \item Suba un archivo \texttt{.zip} con la estructura completa al aula virtual.
\end{itemize}

\end{document}
